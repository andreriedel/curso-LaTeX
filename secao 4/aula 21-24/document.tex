\documentclass[12pt]{report}

\usepackage[utf8]{inputenc}
\usepackage[brazil]{babel}

\usepackage{indentfirst}
\usepackage{setspace}
\usepackage[a4paper, left=3cm, top=3cm, right=2cm, top=2cm]{geometry}
\usepackage[usenames, dvipsnames]{xcolor}

\usepackage{graphicx}
\usepackage{float}

\usepackage{multirow}
\usepackage{tabularx}

\usepackage{amsmath} % adiciona o modo matemático

% ---------------------------------------------------------------------------- %

% *ANCHOR tags globais

\setlength{\parindent}{1.5cm}
\setlength{\parskip}{0.5cm}

\renewcommand{\sin}{\mathrm{sen\hspace{0.5mm}}} % renomeia 'sin' para 'sen'
\renewcommand{\tan}{\mathrm{tg\hspace{0.5mm}}} % renomeia 'tan' para 'tg'

% ---------------------------------------------------------------------------- %

\begin{document} % *ANCHOR início do documento

\section{Modo matemático}

Esta é uma equação do segundo grau: $ax^2 + bx + c = 0$. % equação inline
A solução dessa equação é dada pela fórmula de Bhaskara:

\begin{equation} % equação matemática inblock
    x = \frac{-b \pm \sqrt{b^2 - 4ac}}{2a}
\end{equation}

\begin{equation*} % equação não numerada
    % uso de um vetor para escrever múltiplas equações na mesma linha:
    \begin{array}{cc}
        x_1 = \dfrac{-b - \sqrt{b^2 - 4ac}}{2a}, &
        x_2 = \dfrac{-b + \sqrt{b^2 - 4ac}}{2a}
    \end{array}
\end{equation*}

\begin{equation*}
    A =
    \begin{bmatrix}
        1 & 0 & 0 \\
        0 & 1 & 0 \\
        0 & 0 & 1
    \end{bmatrix}
\end{equation*}

\begin{equation*}
    \begin{matrix}
        \sin{2x} \\
        \tan{2x}
    \end{matrix}
\end{equation*}

\begin{equation*}
    \left(\frac{2a}{3b}\right)
\end{equation*}

\begin{equation*}
    \{2a\}
\end{equation*}

\begin{equation*}
    x_{12}
\end{equation*}

\end{document}
